\section{General description}

The selected collection represents a personal photo library, containing about
3600 image files. As the feedback for part one of the assignment stated, we
concentrated on just JPG-images (rather than also including video files).\\

However, as we tried to select samples as broad as possible, we included samples
from other filetypes in part one of our assignment. We had to dismiss these
samples in order to move on to part two. This reduced our number of samples to
5.\\

Note that we were not able to upload two of the result sample files because of
their size (larger than 20 MiB).

\section{Alternative selection}

As our selected file format was JPG, we had to decide what image format to use
for digital preservation. One alternative was to simply keep the status quo.
However, in order to produce more meaningful results, we opted for converting
the files. Another reason for switching from JPG is the fact that there are
still patent lawsuits active that concern the JPG format \cite{digpres:2013}.\\

For alternative formats, we quickly decided to select the following formats for
further evaluation:

\begin{itemize}
    \item TIFF
    \item BMP
\end{itemize}

For tool selection, we decided to use the popular free open source image
manipulation toolkit \texttt{ImageMagick}, as it provides a handy command-line
interface that eases automation.\\

As an alternative to \texttt{ImageMagick} we also selected
\texttt{GraphicsMagick}, an \texttt{ImageMagick} clone with supposedly better
performance.

\section{Experiments}

We decided to write little shellscrips for each of the selected tools. Listing
\ref{lst:preserve-sh} depicts the shellscript for preserving a directory tree
via \texttt{ImageMagick} (the \texttt{GraphicsMagick} variant is very similar,
just the actual call for the conversion is different).

\begin{lstlisting}[caption=Shellscript for preserving a directory tree via
ImageMagick, label=lst:preserve-sh]
bla
asdf
qwer
for bla
    todo shellscript here
asdf
\end{lstlisting}

A sample invocation here looks as follows:

\begin{lstlisting}
> ./preserve.sh /path/to/source /path/to/destination tiff
\end{lstlisting}

This will recreate the source directory tree at the specified destination, with
each JPG-file being converted to the target format (in this case TIFF).\\

These shellscripts enabled a high degree of automation for performing our
experiments. In order to process all of our samples quickly, a custom directory
tree was arranged that contained nothing but our sample files (but at their
correct relative location to each other).\\

Each invocation of the scripts was timed with the gnu utility \texttt{time}.

\section{Conclusions}

In the end PLATO recommended the conversion to TIFF files. 

\section{PLATO Feedback}
